\documentclass[11pt]{article}
\begin{document}

\noindent Ott, J. Strategies for characterizing highly polymorphic markets in human gene mapping,
Am J Hum Genet 51: 283-290, 1992
   
\bigskip\noindent Xiong, Momiao and Jin, Li. Combined linkage and linkage disequilibrium mapping for genome
screens, Genet Epidemiol 19: 211-234, 2000

\bigskip\noindent Huang, Jian and Jiang, Yanming, The score statistic of the {LD-Lod} analysis: detecting
linkage adaptive to linkage disequilibrium, Hum Hered 52: 83-98, 2001

\bigskip\noindent Zhao, L. P. and Hsu, L. and Davidov, O. and Potter, J. and Elston, R. C. and Prentice, R.  
L., Population-based family study designs: An interdisciplinary research framework for Genetic Epidemiology,
Genet Epidemiol 14:365-388, 1997


\bigskip\noindent{\large Example from Terwilliger \& Ott (1994) and some experiments on linkage and LD 
analysis}

\bigskip\noindent Initial haplotype frequencies were obtained from founder individuals using {\bf
EH}, and the input data, not given there, is as follows.
\begin{verbatim}
2 2
3 0 0
1 8 0
0 1 0
\end{verbatim}
Note that the first line indicates that there are two markers each with two
alleles, the following lines correspond to a $3\times 3$ table formed by the
three genotypes at each marker.

Denote the allele frequencies at marker 1 to be $p_1, p_2$ and those at marker
2 to be $q_1, q_2$.  A single disequilibrium parameter ($\delta$) is sufficient
to describe disequilibrium between two biallelic loci.  The variation between
$\delta=0$ and $\delta=-p_1q_1$ or $p_2q_2$ indicates no and complete linkage
disequilibrium.  This motivates us to incorporate this parameter into the
standard linkage analysis, so that the likelihood calculation is based directly
on haplotype frequencies and a proper optimisation procedure is used.

Consider a generalised single locus with allele frequencies $p$ and $q$ for the
normal and disease alleles, respectively. The penetrances of genotypes given
two, one and no disease alleles are $f_2$, $f_1$ and $f_0$.  Consider also a
marker locus with $n$ alleles, their associated haplotype frequencies being
$h_{11},...,h_{1n}, h_{21},...,h_{2n}$, conditional on individual's affection
status, the genotypic distribution is then obtained according to equation
in Sham (1998).  Alzheimer's model (Model 3 below) is used to define genotypic
probabilities, through which 22 models are generated (see table~\ref{models6})
by a SAS program.

\begin{table}[h]
\caption{The 22 models compatible with Alzheimer's\label{models6}}
\centering
\vskip 0.3cm
\begin{tabular}{llllll}
\hline
model &   $q$  & $\delta$ &    $f_0$ &    $f_1$  &    $f_2$\\
\hline
  1   & 0.117  & 0.10027  & 0.049410 &  0.21760  &  0.95834\\
  2   & 0.130  & 0.10179  & 0.045761 &  0.20521  &  0.92021\\
  3   & 0.130  & 0.11310  & 0.050000 &  0.20000  &  0.80000\\
  4   & 0.143  & 0.10365  & 0.042493 &  0.19338  &  0.88008\\
  5   & 0.156  & 0.09454  & 0.034453 &  0.18342  &  0.97647\\
  6   & 0.156  & 0.10585  & 0.039598 &  0.18224  &  0.83867\\
  7   & 0.169  & 0.09707  & 0.031942 &  0.17163  &  0.92221\\
  8   & 0.182  & 0.09994  & 0.029809 &  0.16087  &  0.86821\\
  9   & 0.195  & 0.09184  & 0.022419 &  0.14611  &  0.95217\\
 10   & 0.195  & 0.10315  & 0.028020 &  0.15114  &  0.81520\\
 11   & 0.208  & 0.09539  & 0.021137 &  0.13677  &  0.88495\\
 12   & 0.221  & 0.08797  & 0.014451 &  0.11813  &  0.96561\\
 13   & 0.221  & 0.09928  & 0.020143 &  0.12860  &  0.82105\\
 14   & 0.234  & 0.09220  & 0.013983 &  0.11129  &  0.88580\\
 15   & 0.247  & 0.08545  & 0.008373 &  0.08962  &  0.95922\\
 16   & 0.247  & 0.09676  & 0.013721 &  0.10557  &  0.81225\\
 17   & 0.260  & 0.09035  & 0.008553 &  0.08621  &  0.86903\\
 18   & 0.273  & 0.08428  & 0.004232 &  0.06279  &  0.93176\\
 19   & 0.286  & 0.08986  & 0.004807 &  0.06335  &  0.83481\\
 20   & 0.299  & 0.08446  & 0.001783 &  0.03972  &  0.88456\\
 21   & 0.312  & 0.07940  & 0.000161 &  0.01230  &  0.93739\\
 22   & 0.325  & 0.08599  & 0.000580 &  0.02182  &  0.82145\\
\hline
\end{tabular}
\end{table}

For example with model 1, the haplotype frequencies are as follows \medskip

\begin{tabular}{llll}
\\
                &  \multicolumn{2}{c}{marker}\\ %\cline{2-3}
disease         &   allele 1 & allele 2 & total\\
disease allele  & 0.115479 &  0.014521 &   0.13\\
normal  allele  & 0.001521 &  0.868479 &   0.87\\
total           & 0.117    &  0.883    &   1\\
\\
\end{tabular}
\medskip

Only a subset of these models, here models 1, 3, 11 and 22 are used to
experiment on model identification.  The second marker is assumed to have
three, four, or five equifrequent alleles and in linkage equilibrium with the
first (allele frequency 0.1).  Nuclear families are generated from the {\bf
SIM} program (unpublished), with sibship size in accordance with truncated
Poisson distribution with parameter 3.  Families are kept only when there are
two or more family members are affected.  For the simulated data, numerical
optimisation is conducted to recover the simulated parameters.  If $q$ is
fixed, it is set to 0.13.  If $\delta$ is fixed, it is set to be the maximum
0.1311.  -2 log likelihood is minimised over penetrances in addition to
different combination of frequency and disequilibrium parameters.  The problem
is redefined as follows.

When $f_0=f_1$ and $f_2=1$, $f_1$ achieves the smallest value
${[K-q^2]}/{[1-q^2]}$, $f_1\ge \max({[K-q^2]}/{[1-q^2]},0)$, and when $f_0=0$
and $f_1=f_2$ the biggest ${K}/{[1-(1-q)^2]}$,
$f_1\le\min({K}/{[1-(1-q)^2]},1)$.

Given $f_1,q$, if $f_1<K$, $f_2$ achieves the smallest value when $f_0=f_1$,
i.e. ${[K-(1-q^2)f_1]}/{q^2}$; If $f_1>K$, the smallest $f_2$ is $f_1$, the
largest is when $f_0=0$, i.e.  $\min({[K-2q(1-q)f_1]}/{q^2},1)$.

Given $q, f_1, f_2$, $f_0={[K-2q(1-q)f_1-q^2f_2]}/{(1-q)^2}$.

Given the lower bound $L$ and upper bound $U$ of a parameter $x$, the problem
is then converted into a nonlinear optimisation with boundary constraints.
$x=L+c(U-L),$ where $c$ varies between 0 and 1.
\begin{eqnarray*}
 \min f_1&=&\max((K-q^2)/(1-q^2),0)\cr
 \max f_1&=&\min(K/(1-(1-q)^2),1)\cr
 f_1&=&\min f_1+c_1(\max f_1-\min f_1)\cr \cr
 \min f_2&=&\cases{f_1, & $f_1>K$ \cr  K-(1-q^2)f_1)/q^2,& otherwise}\cr\cr
 \max f_2&=&\min((K-2qpf_1)/q^2,1)\cr
 f_2&=&\min f_2+c_2(\max f_2-\min f_2)\cr
 f_0&=&(K-2qpf_1-q^2f_2)/p^2\cr
 \min\delta&=&\max(-qp_1,-pp_2)\cr
 \max\delta&=&\min(pp_1,qp_2)\cr
 \delta&=&\min\delta+c_3(\max\delta-\min\delta)\cr
 K_n&=&p^2f_0 + 2pqf_1 + q^2f_2
\end{eqnarray*}
Where $c_1-c_3$ are free parameters for the penetrances and disequilibrium
parameter, where $K_n$ is used to check for $K$.  Logistic transformation
$f(x)={1}/{[1+\exp(-x)]}$ removing the boundary constraints is an alternative
but not pursued.

The optimisation is performed with the conjugate gradient method of SAS by
repeatedly invoking {\bf LINKMAP}.  Statistical power can be appropriately
evaluated in terms of sample sizes based on the log-likelihood ratio
statistics.  For type I and type II error rates to be $\alpha=0.0001$,
$\beta=0.10$, the noncentrality parameter is 29.92 for $\chi^2$ degrees of
freedom 2.

Based on Figure in Terwilliger \& Ott, the likelihood ratio statistic of association
versus no association is 12.02, corresponding to a $p$ value of 0.0005.  The
estimated haplotype frequencies are (0.377219, 0.199704, 0.276627, 0.146450)
assuming independence and (0.576921, 0.000002, 0.076925, 0.346152) assuming
association; these yield a disequilibrium estimate of 0.199702.

These are used as initial values for {\bf ILINK} to maximise the likelihood
over haplotype frequencies using the whole pedigree to establish the phase.
When the markers are unlinked the estimate (0.576837, 0.000000, 0.076995,
0.346169) is very similar to those from {\bf EH}.  By allowing for linkage
between the two markers, and estimate the recombination and haplotype
frequencies jointly the results become (0.315367, 0.257574, 0.341849, 0.085210)
and disequilibrium value of -0.061852.  The values of -2 log-likelihood from
{\bf ILINK} are 92.9909 and 97.9948120 so the optimisation process is not
satisfactory.  Such a seemingly simple example shows that the likelihood
surface is not simple.

The same result occurs from SAS optimisation procedure:  fixing recombination
rate $\theta=0.5$ and starting with equal haplotype frequencies, -2
log-likelihood value drops from 108.13096 to 92.991507 with estimates
(0.5759099, 0, 0.0782231 and 0.345867) under Sun, -2 log-likelihood 85.335481
and haplotype estimates (0.6622288, 0.0000000, 0.1902562, 0.1475151) under DEC
Alpha, while allowing for linkage, -2 log-likelihood remains to be 97.975395
with haplotype frequency estimates (0.3168155, 0.2601083, 0.3370297, 0.0860466)
under DEC Alpha, suggesting a very flat surface with respect to $\theta$.  It
seems the failure of SAS may be due to convergence to a local minimum, but log
likelihood remains unchanged with moderate change in parameter value.

The result is not examined in detail here for the simulated data from {\bf
SIM} due to some vexing problems.  First, the scheme from which the sample is
simulated may lead to ascertainment bias.  Second, it discards families
potential for LD, such as family trios (two parents and one child), which is
typical data for TDT designs for test of linkage and association.  Finally, the
implementation in {\bf LINKAGE} is quite limited.

The lod score function indeed favors larger disequilibrium value.  In general,
let $S_{ijkl}$ be the probability of parental genotype $(i,j)\times (k,l)$) at
a multiallelic marker, $S_{ijkl}(\theta, 0)=p_ip_jp_kp_l$ under linkage
equilibrium; $S_{ijkl}(\theta,\delta)=p_ip_jp_kp_lW(\theta,\delta)$ under
linkage disequilibrium ($W(\theta,\delta)$ is a function of disease/marker
allele frequencies and disequilibrium values between them).  The noncentrality
parameter for family trios (two parents and an affected sibling, called S
families) is $$\lambda_S(\theta,\delta)=4N
\sum_{ijkl}p_ip_jp_kp_lW(\theta,\delta)\ln W(\theta,\delta)$$ For single
mutation, and from $\delta(t)=\delta_0e^{\theta t}$, with $t$ being the number
of generations that a single mutation was introduced, they obtained the
approximation $$\lambda_S(\theta,\delta) \approx
\frac{4N[\theta^2+(1-\theta)^2]}{p^2(A)}\left[ (f_{DD}-f_{Dd})p_D
+(f_{Dd}-f_{dd})p_d\right]\frac{1-p_1}{p_1} e^{-2\theta t}$$ Where $p$'s and
$p_D$ are the marker and disease allele frequencies.
$P(A)=\sum_{r}\sum_{t}f_{rt}p_rp_t$, $r,t=d,D$.  
It was surprising that all the 22 models
virtually yielded the same likelihood and noncentrality parameter, which needs
to be further investigated.

\end{document}
