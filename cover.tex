% -*-latex-*-
% $Log: cover.tex,v $
% Revision 1.3  93/05/17  17:06:29  starflt
% Added acknowledgements section (suggested by tompalka)
%
%
% Revision 1.2  92/04/22  13:13:13  epeisach
% Fixes for 1991 course 6 requirements
% Phrase "and to grant others the right to do so" has been added to
% permission clause
% Second copy of abstract is not counted as separate pages so numbering works
% out
%
% Revision 1.1  92/04/22  13:08:20  epeisach

\title{Statistical Power Analysis and Related Issues in Human Genetic Linkage and Association}

\author{Jing Hua Zhao}
\department{Division of Psychological Medicine}
% If the thesis is for two degrees simultaneously, list them both separated by
%% \and like this:
% \degree{Doctor of Philosophy \and Master of Science}
\degree{Doctor of Philosophy}
\prevdegrees{Bachelor of Medicine, Shandong Medical University (1985)\\
            Master of Medicine, Shanghai Medical University (1988)}
\degreemonth{May}
\degreeyear{2004}
\thesisdate{May, 2004}

%% By default, the thesis will be copyrighted to MIT.  If you need to
%% copyright the thesis to yourself, just specify the `vi' documentstyle
%% option.  If for some reason you want to exactly specify the copyright
%% notice text, you can use the \copyrightnoticetext command.
%\copyrightnoticetext{\copyright ~King's College London, 2004}

% If there is more than one supervisor, use the \supervisor command once for
%% each.
\supervisor{Pak Chung Sham}{Professor}

% this is the department committee chairman, not the thesis committee chairman
\chairman{Faculty of Medicine}{Board of Studies Genetics}

% Make the titlepage based on the above information.  If you need something
% special and can't use the standard form, you can specify the exact text of
% the titlepage yourself.  Put it in a titlepage environment and leave blank
% lines where you want vertical space. The spaces will be adjusted to fill
% the entire page. The dotted lines for the signatures are made with the
% \signature command.
\maketitle

% The abstractpage environment sets up everything on the page except the
% text itself.  The title and other header material are put at the top
% of the page, and the supervisors are listed at the bottom.  A new page
% is begun both before and after.  Of course, an abstract may be more
% than one page itself.  If you need more control over the format of the
% page, you can use the abstract environment, which puts the word
% "Abstract" at the beginning and single spaces its text.

%% You can either \input (*not* \include) your abstract file, or you can put
%% the text of the abstract directly between the \begin{abstractpage} and
%% \end{abstractpage} commands.

% First copy: start a new page, and save the page number.
\newpage
\pagestyle{empty}
\setcounter{savepage}{\thepage}
\begin{abstractpage}
The difficulty with genetic study of complex traits has raised concerns over
optimal study design and statistical analysis.  The key statistical issues in
designing different studies are validity, power and robustness of relevant
statistical tests.  This thesis investigates several scenarios of power
analysis in linkage and association analysis, which include linkage tests in
small pedigrees, association tests for case-control data, marker polymorphism
and mutation detection, computer simulation methods and application.  It also
gives results of numerical experiments on family haplotype analysis and
discusses TDT and other association designs.  This thesis reveals that commonly
used parametric and nonparametric linkage statistics are comparable in power
for two point analysis with simple families, but nonparametric linkage
statistics are anticonservative in some families.  As for case-control data,
heterogeneity statistic nearly has power close to the true model without the
needs of disease model specification, and comparable to the ordinary likelihood
ratio test for contingency table.  In mutation detection, multiallelic marker
is usually more favourable than SNP.  While haplotype analysis has claimed
power for linkage and association, there may be numerical and analytical
difficulties when a likelihood approach using family data is adopted.  Finally,
correct sample sizes are obtained for TDT design as reported earlier in the
literature, and some computer routines performing these calculations are also
given.

\end{abstractpage}
\newpage

% Second copy: start a new page, and reset the page number.  This way,
% the second copy of the abstract is not counted as separate pages.
% \newpage
% \setcounter{page}{\thesavepage}
% \begin{abstractpage}
% The difficulty with genetic study of complex traits has raised concerns over
optimal study design and statistical analysis.  The key statistical issues in
designing different studies are validity, power and robustness of relevant
statistical tests.  This thesis investigates several scenarios of power
analysis in linkage and association analysis, which include linkage tests in
small pedigrees, association tests for case-control data, marker polymorphism
and mutation detection, computer simulation methods and application.  It also
gives results of numerical experiments on family haplotype analysis and
discusses TDT and other association designs.  This thesis reveals that commonly
used parametric and nonparametric linkage statistics are comparable in power
for two point analysis with simple families, but nonparametric linkage
statistics are anticonservative in some families.  As for case-control data,
heterogeneity statistic nearly has power close to the true model without the
needs of disease model specification, and comparable to the ordinary likelihood
ratio test for contingency table.  In mutation detection, multiallelic marker
is usually more favourable than SNP.  While haplotype analysis has claimed
power for linkage and association, there may be numerical and analytical
difficulties when a likelihood approach using family data is adopted.  Finally,
correct sample sizes are obtained for TDT design as reported earlier in the
literature, and some computer routines performing these calculations are also
given.

% \end{abstractpage}

\newpage

\section*{Acknowledgments}

I would like to thank Professor Pak Sham and Dr David Curtis for their support
and supervision, which makes it possible for me to conduct this project.  I am
also grateful of Professor Robin Murray for his support for the final year of
my work.

As a six-year tenant in \IoP\ I am indebted to many people.  I wish to
express special thanks to the former personnel manager Ms Catriona Urquart
for efficiently sorting out my visa several times, to Jenny Merchant,
Andrew Wallis, Muriel Walsh and Maureen Armstrong-Smith for their help
with departmental affairs, to Lee Wilding and Debbie Heavey for academic
information, to Malcolm Hart and Eric Glover for their help with
computing, to Professor Peter McGuffin for reading and supporting my
upgrade report, and to Ms Marion Cuddy for reading through final version
of the draft, and to all friends in the institute for their help on many
other matters.  Chapters 1-4 of he thesis were improved at suggestions
from Professors John Whittaker (Imperical College, London) and Tim Bishop
(Cancer Research UK, St James's University Hospital, Leeds). Occasional
revisions to Chapter 2 of the thesis were made while working as a
statistician in the Department of Epidemiology and Public Health,
University College London over a period of about two years, but due to my
new role in the Whitehall II study including more commitment to
longitudinal data analysis and social epidemiology, no more references
have been added since my viva in July 2002. I am grateful of Professor Sir
Michael Marmot and his Whitehall II team for making this possible.

I am grateful of Professors Zhaohuan Zhang and Fumin Shen of Shanghai (Fudan)
Medical University for igniting my interests in genetic epidemiology during my
master training between 1985 and 1988, Professor Xiping Xu of Harvard
University for resuming my work in this area from 1995, Professors Jin Ma and
Yuanli Liu of Harvard University, Professor Pihuan Jin of Shanghai Medical
University for supporting my coming to the UK.  Thanks also to my former
colleagues and friends in China for their concern over so many years.

Finally I would like to thank my special friend, Dr Wendi Qian, for reading
through my upgrade report as well as the first draft of this manuscript and
giving many helpful comments.  I wish to dedicate this thesis to my family for
their patience, love and care.
