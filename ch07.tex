
\chapter{Transmission/disequilibrium tests}

\section{Introduction}

TDT has become a strong alternative to traditional linkage design, owing
particularly to the work of Risch \& Merikangas (1996).  The power of TDT using
both family trios and sibling controls has been investigated by Kaplan et al.
(1997), Baur \& Knapp (1997) Knapp (1999a, 1999b) Cervino \& Hill (2000), among
others.

The work of Risch \& Merikangas (1996) has provoked heated debates and studies
concerning the use of linkage versus association design (Scott et al.  1997;
M\"{u}ller-Myhsok \& Abel 1997; Long et al.  1997; Risch \& Teng 1998; Morton
\& Collins 1998; Teng \& Risch 1999; Long \& Langley 1999; Camp 1997; Knapp
1999a). These are particularly relevant in the context of increasingly dense
genetic map.  When serving as a guideline for designing a TDT study, it is
unfortunate that the paper of Risch \& Merikangas (1996) contained a
programming error.  A correction will shed light on later work such as Camp
(1997, 1999) and represent the original conclusions in terms of sibling instead
of genotypic relative risks. Comparison with that of a case-control design will
further clarify important issues made by those discussants.


\subsection*{Chapter aims}

As a practical implementation of Risch \& Merikangas (1996) while correcting an
error in their program, this chapter examines power of several designs:  ASP
linkage, singleton/sib-pair TDT and case-controls.  The result has been used in
a personal communication to Dr Anthony D. Long (UC Irvine).  In addition, the
relationship between the case-control method and TDT by Mitchell (2000) is
sketched.


\section{Methods}

Assume a disease susceptibility locus has two alleles A and a with population
frequencies $p$ and $q=1-p$, the penetrances to be $f_0$, $f_1$ and $f_2$
(defined as the conditional probabilities of the disease given 0, 1 and 2
copies of the disease allele), and the recombination fraction between the
disease locus and a marker $\theta$.  We have population prevalence of the
disease $K = p^2f^2+2pqf_1+q^2f_0$, additive and dominance variances $V_A =
2pq[p(f_2-f_1)+q(f1-f_0)]^2$, $V_D = p^2q^2(f_2-2f_1+f_0)^2$, $\Psi =
\theta^2+(1-\theta)^2$.  Suarez et al.  (1978) derived IBD probabilities for
affected sib-pair as follows \begin{eqnarray*} P(IBD=0) & = &\frac{1}{4}-
\frac{(\Psi-.5)V_A+(2 \Psi - \Psi^2 - .75)V_D} {4(K^2 + .5V_A + .25V_D)} \cr
P(IBD=1) & = &\frac{1}{2}- \frac{2(\Psi^2-\Psi+.25)V_D} {4(K^2 + .5V_A +
.25V_D)} \cr P(IBD=2) & = &\frac{1}{4}+ \frac{(\Psi-.5)V_A+(\Psi^2 - .25)V_D}
{4(K^2 + .5V_A + .25V_D)} \end{eqnarray*} Under no linkage, the probabilities
of affected sib pair sharing 0, 1, and 2 alleles IBD are 1/4, 1/2, and 1/4.
Under linkage these sharing probabilities will be different.

Now assume the genotypic relative risks for genotypes AA, Aa, aa are
$\gamma^2$, $\gamma$ and $1$.  The population disease prevalence, the additive
and dominance variances are therefore
$K=p^2\gamma^2+2pq\gamma+q^2=(p\gamma+q)^2$, $V_A=2pq(\gamma-1)^2(p\gamma+q)^2$
and $V_D=p^2q^2(\gamma-1)^4$.  The offspring and sibling relative risks
$\lambda_O=1+{0.5V_A}/{K^2}$ and $\lambda_S=1+{(0.5V_A+0.25V_D)}/{K^2}$ can be
rewritten as $\lambda_O=1+w$ and $\lambda_S=(1+0.5w)^2$, where
$w={(pq(\gamma-1)^2)}/{(p\gamma+q)^2}$.  Since the probabilities of siblings
sharing none or one allele by descent is $z_0={0.25}/{\lambda_S}$ and
$z_1={0.5\lambda_O}/{\lambda_S}$, the nonshared probability is
$Y=1-0.5z_1-z_0=1-{0.25(\lambda_O+1)}/{\lambda_S}={(1+w)}/{(2+w)}$.  The
probability of a parent of an affected child (Aff Child) being heterozygous (H)
is given by $P(H|\mbox{Aff Child})={P(H)P(\mbox{Aff Child}|H)}/{P(\mbox{Aff
Child})}$, which is ${2pq(0.5p(\gamma^2+\gamma)+0.5q(\gamma+1))}/
{(p\gamma+q)^2}={pq(\gamma+1)}/{(p\gamma+q)}$.

Consider a set of $N$ independent identically distributed random variables
$B_i$ with mean 0 and variance 1 under the null hypothesis, mean mean $\mu$ and
variance $\sigma^2$ under the alternative hypothesis.  Then the statistic
$\sum_{i=1}^N B_i/N$ has mean 0 and variance 1 under the null but mean
$\sqrt{N}\mu$ and variance $\sigma^2$ under the alternative.  The sample size
$N$ for a given significance level $\alpha$ and power $1-\beta$ can be
estimated by $(Z_\alpha-\sigma Z_{1-\beta})^2/\mu^2$.

For affected sib-pair linkage analysis, the allele shared and nonshared from
the $i$th parent is a random variable denoted by $B_i, i=1,\ldots,N$ and scored
$1$ and $-1$.  Under the null hypothesis, the shared and nonshared each has
probability 0.5 so the mean and variance of $B_i$ are 0 and 1.  Under the
alternative $\mu=2Y-1$ and $\sigma^2=4Y(1-Y)$.  Assuming sharing of alleles
from both parent to be independent, the required sample size for affected
sib-pair under $\theta=0$ and no linkage disequilibrium is $N={(z_\alpha-\sigma
z_{1-\beta})^2}/{2\mu^2}$, $Y={(1+w)}/{(2+w)}$,
$w={pq(\gamma-1)}/{(p\gamma+q)^2}$ as above.  As for singleton and sib-pair TDT
with disease locus or a nearby locus in complete disequilibrium, the number of
transmissions of allele A is scored from heterozygous parents.  For singleton,
denote $B_i=1/\sqrt{h}, h={pq(\gamma+1)}/{(p\gamma+q)}$ if parent is
heterozygous and transmits A, $B_i=0$ if parent is homozygous,
$B_i=-1/\sqrt{h}$ if parent is heterozygous and transmits a.  Under the null
hypothesis the mean and variance of $B_i$ are 0 and 1, whereas under the
alternative they are $\sqrt{h}{(\gamma-1)}/{(\gamma+1)}$ and
$1-h[{(\gamma-1)}/{(\gamma+1)}]^2$, respectively.  When sib-pairs instead of
singletons are used in TDT analysis, the probability of parental heterozygosity
becomes $h={pq(\gamma+1)^2}/{[2(\gamma p+q)^2+pq(\gamma-1)^2]}$, the same
formula for sample size calculation can be applied and the required number of
families is half the expected number since there are two independent affected
sibs.

This can be compared with those of case-control design using a statistic
directly testing association between marker and disease (Long et al.  1997).
Suppose we have a randomly ascertained population sample, under HWE
multiplicative model in which the genotypic relative risk is $\gamma$, the
frequencies of the three disease genotypes AA, Aa and aa in cases are
$\pi\gamma^2$, $2\pi\gamma pq$, and $\pi q^2$, respectively, where $\pi$ is the
``baseline'' probability that an individual with $aa$ genotype being affected.
Similarly the three frequencies in controls are $(1-\pi\gamma^2)p^2$,
$2(1-\pi\gamma)pq$ and $(1-\pi)q^2$.  A unit $\chi^2$ statistic can be
constructed using the following $2\times 2$ table:

\medskip
\begin{center}\begin{tabular}{c|cc|l}
  & affected genotype & nonaffected genotype & \\
\hline
A & $\pi\gamma^2p^2+\pi\gamma pq$ & $(1-\pi\gamma^2)p^2+(1-\pi\gamma)pq$ &$p$\\
a & $\pi\gamma pq+\pi q^2$ & $(1-\pi\gamma)pq+(1-\pi)q^2$ & $q$ \\
\hline
  & $\pi(\gamma p+q)^2$ & $1-\pi(\gamma p+q)^2$ & $1$ \\
\end{tabular}\end{center}

\medskip or equivalently\medskip

\begin{center}\begin{tabular}{c|cc|l}
& affected genotype & nonaffected genotype \\ \hline
A & $\pi\gamma p(\gamma p+q)$ & $p-\pi\gamma p(\gamma p+q)$  & $p$ \\
a & $\pi q(\gamma p+q)$ & $q-\pi q(\gamma p+q)$ & $q$\\ \hline
  & $\pi(\gamma p+q)^2$ & $1-\pi(\gamma p+q)^2$ & $1$\\
\end{tabular}\end{center} \medskip

So that the expected unit frequencies (E) are $\pi p(\gamma p+q)^2$, $\pi
q(\gamma p+q)^2$, $p-\pi p(\gamma p+q)^2$ and $q-\pi q(\gamma p+q)^2$, the
discrepancies between observed and expected frequencies all have factor $\pi
pq(\gamma p+q)(\gamma-1)$ but with negative sign before the second and the
third items.  The statistic $X^2=\sum (O-E)^2/E$ is then easy to obtain.  Under
the null hypothesis $\gamma=1$, $X^2$ has central $\chi^2$ distribution,
whereas under the alternative $\gamma>1$ and $X^2$ follows noncentral
$\chi_{1,\delta}^2$ distribution with noncentrality parameter $\delta=[\pi
pq(\gamma-1)^2]/[1-\pi (\gamma p+q)^2]$ or $\delta=[\gamma^2 p+q-(\gamma
p+q)^2]/[1-\pi (\gamma p+q)^2]$ It is equivalently to derive the power by
$Y=\sqrt{X^2}\sim N(\sqrt{\delta},1)$.  $1-\beta=\Phi(-Z<Y<Z)$, where $Z$ is
a preassigned standard normal deviate.

Let the genome-wide significance level and type II error rate be
$\alpha=5\times 10^{-8}$ and $\beta=0.2$, respectively, the power is calculated
for random ascertainment with three different disease prevalences.

A C program written to implement the above procedure is listed in
Appendix~\ref{programs}.


\section{Results}

The results are shown in Table~\ref{table7_1}.  Column $N_L$ corrects the
calculation from the original paper (as $N_{asp}(\times)$).  The Alzheimer's
disease model is based on Scott et al.  (1997).

\begin{table}[h]
\centering
\caption{Comparison of linkage and association in nuclear
families required for identification of disease gene\label{table7_1}}
\vskip 0.3cm
\begin{tabular}{cccccccccc}
\hline
&&\multicolumn{2}{c}{Linkage}&&\multicolumn{4}{c}{Association} \\
\cline{3-4} \cline{6-9}
$\gamma$& $p$ &$Y$ & $N_{L}$
& $P_A$& $Het_1$ &  $N_{tdt}$ & $Het_2$ &$N_{asp/tdt}$ & $N_{asp}(\times)$
\\
\hline
\\
 4.0&0.01& 0.520&     6400& 0.800& 0.048&   1098&0.112&    235&       4260\\
 &0.10& 0.597&      276& 0.800& 0.346&    150&0.537&     48&        185\\
 &0.50& 0.576&      445& 0.800& 0.500&    103&0.424&     61&        297\\
 &0.80& 0.529&     3022& 0.800& 0.235&    222&0.163&    161&       2013\\
\\
 2.0&0.01& 0.502&   445835& 0.667& 0.029&   5823&0.043&   1970&     296710\\
 &0.10& 0.518&     8085& 0.667& 0.245&    695&0.323&    264&       5382\\
 &0.50& 0.526&     3751& 0.667& 0.500&    340&0.474&    180&       2498\\
 &0.80& 0.512&    17904& 0.667& 0.267&    640&0.217&    394&      11917\\
\\
 1.5&0.01& 0.501&  6943229& 0.600& 0.025&  19320&0.031&   7776&    4620807\\
 &0.10& 0.505&   101898& 0.600& 0.214&   2218&0.253&    941&      67816\\
 &0.50& 0.510&    27040& 0.600& 0.500&    949&0.490&    484&      17997\\
 &0.80& 0.505&   101898& 0.600& 0.286&   1663&0.253&    941&      67816\\
\\
\multicolumn{3}{c}{Alzheimer's:}\\
\\
 4.5&0.15& 0.626&      163& 0.818& 0.460&    100&0.621&     36&        109\\
\\
\hline
\end{tabular}
\flushleft{$\gamma$=genotypic risk ratio; $p$=frequency of disease allele A;
$Y$=probability of allele sharing; $N_{L}$=number of ASP families required for
linkage; $P_A$=probability of transmitting disease allele A;
$Het_1$,$Het_2$=proportions of heterozygous parents; $N_{tdt}$=number of family
trios; $N_{asp/tdt}$=number of ASP families}
\end{table}

It turns out that with $\gamma\le 2$, the expected marker-sharing only
marginally exceeds 50\% for any allele frequency ($p$).  The use of linkage
would need practically nonachievable sample size.  Nonetheless, direct tests of
association with a disease locus itself can still be quite strong.  But it may
involve large amount of statistical testing of associated alleles.

The result of the case-control design is shown in Table~\ref{ccpower}.  Long et
al.  (1997) used approximation $1-\beta=\Phi(Z-\sqrt{\delta})$ by taking the
area under normal curve in the lower tail as negligible.  It seems this
approximation is quite good (Columns 3-5).

\begin{table}[h]
\centering
\caption{Estimated sample sizes required for association detection
\label{ccpower}}
\begin{tabular}{cccccccc}
\\
\hline
& & \multicolumn{3}{c}{Long et al. (1997)} & \multicolumn{3}{c}{Actual calculation}\\
\cline{3-5}\cline{6-8}
$\gamma$ & $p$   &  1\%   &5\%   &  10\%  &       1\% &      5\% &      10\% \\
\hline
   4.0   & 0.01  & 46681  &8959  & 4244   &      46637&      8951&      4240\\
         & 0.10  & 8180   &1570  & 744    &       8172&      1568&       743\\
         & 0.50  & 10890  &2090  & 990    &      10880&      2088&       989\\
         & 0.80  & 31473  &6040  & 2861   &      31444&      6035&      2859\\
\\
   2.0   & 0.01  & 403970 &77530 & 36725  &     403593&     77457&     36690\\
         & 0.10  & 52709  &10116 & 4792   &      52660&     10106&      4787\\
         & 0.50  & 35284  &6772  & 3208   &      35252&      6765&      3205\\
         & 0.80  & 79390  &15236 & 7217   &      79316&     15222&      7211\\
\\
   1.5   & 0.01  & $>10^6$ &307055& 145447 &    1598429&    306769&    145312\\
         & 0.10  & 192104 &36869 & 17464  &     191925&     36834&     17448\\
         & 0.50  & 98012  &18810 & 8910   &      97921&     18793&      8902\\
         & 0.80  & 192104 &36869 & 17464  &     191926&     36834&     17448\\
\hline
\end{tabular}
\end{table}

Clearly it is most favourable for diseases that are relatively common, which
has important implications for complex traits.  When the disease is relatively
common, the disease-allele frequency is intermediate and its effect small,
statistical power comparable to that of standard family-based linkage studies
is achieved with a smaller number of randomly sampled individuals.  While
statistical power in terms of required sample size is important, practicality
also needs to be considered.  For efficiently diagnosed late onset disease such
as non-insulin-dependent diabetes and hypertension, it may not be possible to
type parents for affected sibling studies.  A further point is that the actual
number of individuals genotyped needed would be doubled for linkage, tripled
for singleton, and quadrupled for sib-pair, assuming both parents are genotyped
in a affected offspring study (Long et al.  1997, also from
http://dimitri.ucdavis.edu/association\_study/index.html).  Long et al.  (1997)
noted when there is actual marker-disease data, a Fisher's exact test can be
used in the case of population sample to detect association.


\section{Discussion}

Generalisations of Risch \& Merikangas (1996) have been considered by Camp
(1997, 1999) and Knapp (1999a).  One criticism was that genotypic relative risk
$\gamma$ rather than sibling relative risk $\lambda_s$ was employed--since even
with large $\gamma$, $\lambda_s$ could still be small, this was remedied in
Risch (1997).  Moreover, one should not rely too much on complete
disequilibrium assumption (M\"{u}ller-Myhsok \& Abel 1997).  With the same
genetic model and one biallelic marker with alleles $B$ and $b$ of frequencies
$m$ and $1-m$, they considered the following table,

\begin{center}
\begin{tabular}{c|cc|c}
 & B & b \\ \hline
A & $pm+\delta$ & $p(1-m)-\delta$ & $p$ \\
a & $(1-p)m-\delta$ & $(1-p)(1-m)+\delta$ & $1-p$ \\
\hline
  & $m$ & $1-m$ & 1\\
\end{tabular}
\end{center}

\noindent where the linkage disequilibrium parameter $\delta=P(AB)-pm$ achieves
its maximum when $P(AB)$ is $\min(m,p)$.  Let
$\alpha_1=P(A|B)=(pm+\delta)/m=p+\delta/m$ and
$\alpha_2=P(A|b)=[p(1-m)-\delta]/(1-m)=p-\delta/(1-m)$, for TDT with trios the
probability that $Bb$ parent transmits $B$ to the affected child is $\tau (B)=
P(\mbox{aff}|B)/[P(\mbox{aff}|B)+P(\mbox{aff}|b)]$, the prior probabilities of
transmitting $B$ and $b$ are both 0.5, and
$P(\mbox{aff}|B)=[\gamma\alpha_1+(1-\alpha_1)]D$ and
$P(\mbox{aff}|b)=[\gamma\alpha_2+(1-\alpha_2)]D$, where $D$ is the probability
that is a subject is affected given he carries allele $a$.  It turns that out
$\tau (B)=[1+(\gamma-1)\alpha_1]/[2+(\gamma-1)(\alpha_1+\alpha_2)]$, which only
reduces to $\gamma/(1+\gamma)$ when $m=p$ whereas when $m/p$ departs from unity
the power diminishes substantially.  Abel \& M\"{u}ller-Myhsok (1998) also
expressed TDT as $\Lambda=2n\ln[q\ln(q)+(1-q)\ln(1-q)-\ln(0.5)]$, $n$ being the
number of heterozygous parents, $q$ being the probability Bb transmitting B,
and considered the difference between $\Lambda$ and the classic TDT
$d(q)=2n[q\ln(q)+(1-q)\ln(1-q)-\ln(0.5)-2(q-0.5)^2]$.  Assuming that under the
null hypothesis $q\sim N(\tau(B),\sigma^2)$, $\sigma^2=\tau(B)(1-\tau(B))/n$,
the number of heterozygous parents is obtained by solving the following
equation
$$2n[q_\beta\ln(q_\beta)+(1-q_\beta)\ln(1-q_\beta)-\ln(0.5)]=Z_\alpha^2$$ where
$q_\beta=\tau(B)-\sigma Z_{1-\beta}$.  From $n$, the required number of
families $N=n/2h$ can be estimated, $h$ is the probability that a parent with
an affected child is heterozygous given by Risch \& Merikangas (1996), or by
M\"{u}ller-Myhsok \& Abel (1997) as
$u/[u+m^2[1+(\gamma-1)\alpha_1]+(1-m)^2[1+(\gamma-1)\alpha_2]$ with
$u=m(1-m)[2+(\gamma-1)(\alpha_1+\alpha_2)]$.  Setting $\gamma=4$,
$\alpha=5\times 10^{-8}$ and $1-\beta=0.80$, for the maximum likelihood test of
$p=0.5$ (ML-TDT), $N=139$ for $p=m=0.10$ and $N=96$ for $p=m=0.50$, compared
with 150 and 103 in Table~\ref{table7_1}.

A thorough power analysis of various tests would be beyond this chapter.
However a number of observations can be made.  First, TDT is by no means a
substitute for other epidemiological, linkage and case-control association
designs.  Second, the basic principle should apply to both quantitative and
qualitative traits.  Third, it is not limited to nuclear families.  Fourth,
multiply linked marker can preferably be incorporated for a haploptype
disequilibrium test.  Fifth, there is no reason why TDT analysis should be
separated from other analyses.

Ideally, unrelated individuals data from a population, as well as family data,
with information from multiple markers, can be combined into a unified
framework and subject to analyses that are powerful, robust and fast.  For
example, methods as in {\bf TRANSMIT} is very appealing but sometimes it may be
too slow when the Monte Carlo option is used.


\section{Bibliographic notes}

Mitchell (2000) obtained a simple relation between case-control and TDT.  In
her notation, the magnitude of disease-marker association detected by TDT can
be estimated by binomial proportion $T=M_1/(M_1+M_2)$, for the number of times
high ($M_1$) and low ($M_2$) risk alleles are transmitted, respectively.  Let
$m$=allele frequencies of high risk marker allele, $M_1$, $g_i$=P(genotype
i$|$aff)P(aff)/P(genotype i), i=2,1,0 for $M_1M_1,M_1M_2,M_2M_2$.  This
statistic can be rewritten as a function of the relative frequencies of the
marker genotypes among the affected offspring of a heterozygous parent as
follows,

\begin{tabular}{lllll}
\\
Child   &Allele\\
genotype&inherited &   P(genotype)   &P(aff$|$genotype) & Frequencies\\
$M_1M_1$ &   $M_1$ &     $0.5m    $  & $g_2$          & $0.5m(g_2)$\\
$M_1M_2$ &   $M_1$ &     $0.5(1-m)$  & $g_1$          & $0.5(1-m)(g_1)$\\
$M_1M_2$ &   $M_2$ &     $0.5m    $  & $g_1$          & $0.5m(g_1)$\\
$M_2M_2$ &   $M_2$ &     $0.5(1-m)$  & $g_0$          & $0.5(1-m)(g_0)$\\
\\
\end{tabular}

Now let
$A=0.5m(g_2)+0.5(1-m)(g_1)$,
$B=0.5m(g_1)+0.5(1-m)(g_0)$,
then $T=A/(A+B)$, $T$ could be referred to power/sample size formula for
proportion.

Some examples in Mitchell (2000) are as follows,

\begin{tabular}{lll}
\\
Disease           & Marker    &T[case-control, TDT (C.I.)]\\
Cleft lip/palate  & TGFA      &0.67  0.77 (0.63-0.91)\\
Cleft palate      & MSX1      &0.59  0.68 (0.51-0.85)\\
Spina bifida      & 5,10 mthfr&0.60  0.56 (0.49-0.63)\\
IDDM              & 5'FP      &0.77  0.63 (0.54-0.72)\\
\\
\end{tabular}

The estimate of $T$ is outside confidence interval (C.I.) predicted by TDT only
in the last example.  For the first disorder, $T_1$=0.62, $T_0$=0.5, type I
error $\alpha$=0.05, type II error $\beta$=0.20, then the required sample size
becomes
$N=[1.64[0.5(1-0.5)]^{0.5}+0.84[0.62(1-0.62)]^{0.5}]^2/(0.62-0.50)^2\approx
105$, or $N$=266 for $\alpha$=0.001.

McGinnis (1998, 2000) further clarified the relative power of TDT and affected
sib-pair method, including expression of general mode of inheritance.  A
comprehensive evaluation by computer simulation was given by Kaplan et al.
(1997), Monks et al.  (1998).  Others included Baur \& Knapp (1997),
Tr\'{e}gou\"{e}t at al.  (2001).

Cervino \& Hill (2000) compared {\bf TRANSMIT}, {\bf SIBASSOC/STDT} and {\bf
RCTDT} (Knapp 1999b) in a variety of scenarios.  When one or two parents are
missing, the presence of population substructure.  From their simulation, the
classic ``likelihood-ratio association test'' would result in high type I error
when there is a population substructure suggesting simple parameterisation in
terms of transmission and different allele frequencies may not be enough to
characterise the effect of substructure whereas introducing more parameters
would increase the degrees of freedom and reduce power.  It seems {\bf
TRANSMIT} is quite robust to substructure and provides both correct type I
error and reasonable power.  Since {\bf TRANSMIT} is based on a multiplicative
model it is also most powerful when this is true.  {\bf RCTDT} is very
appealing since it combines both vertical and horinzontal approaches, i.e.,
allowing for analysis of incomplete families by reconstructing the missing
parents and by comparing affected and unaffected siblings inside families.

Both Abel \& M\"{u}ller-Myhsok (1998) and Weinberg et al.  (1998) noted that
likelihood ratio test can be more powerful than TDT.  In general, statistical
power of TDT depends on the number of parents heterozygous for particular
alleles.  The magnitude of transmission disequilibrium of a marker allele
depends on factors such as genotype relative risk of the susceptibility locus,
the genetic distance between disease and marker loci, and the amount of linkage
disequibrium (Schaid 1996).  Schaid (1998), Schaid \& Rowland (1998), Schaid
(1999a, 1999b), Schaid \& Rowland (2000), Knapp (1999a, 1999b, 1999c), Horvath
et al.  (2000) gave more accounts about TDT.  A more recent summary of the
statistical issues in TDT was given by Zhao (2000).

An important remark is that case-control design is commonly believed to be more
powerful assuming there is no population stratification (Morton \& Collins
1998; Devlin \& Roeder 1999; Bacanu et al.  2000; Risch 2000; Pritchard \&
Rosenberg 1999; Pritchard et al.  2000; Reich \& Goldstein 2001).  A useful
development was made by Seltman et al.  (2001) which proposed to associate the
evolutionary tree with TDT.  It would be of interests to use the same set of
haplotype frequencies in the Fragile X example to investigate the power of TDT
(Sham \& Curtis 1995a).  A SAS program is available from the section website.
